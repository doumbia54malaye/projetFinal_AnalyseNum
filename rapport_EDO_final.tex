\documentclass{beamer}

\usetheme{Madrid}
\usecolortheme{default}

\title[Analyse numérique]{Comparaison des méthodes numériques\\
Euler, Heun et Runge-Kutta d’ordre 4}
\author{DOUMBIA MALAYE}
\institute{
Discipline : Analyse Numérique\\
Enseignant : Dr ZEZE
}
\date{\today}

\begin{document}

%------------------------------------------------
\begin{frame}
\titlepage
\end{frame}

%------------------------------------------------
\begin{frame}{Introduction et objectif}
La résolution numérique des équations différentielles ordinaires est essentielle
lorsque la solution exacte n’est pas facilement exploitable.

\vspace{0.3cm}
\textbf{Objectif du projet :}
\begin{itemize}
    \item Comparer les méthodes d’Euler, de Heun et RK4
    \item Évaluer leur précision numérique
    \item Comparer leur temps d’exécution
    \item Étudier l’influence du pas de temps
\end{itemize}
\end{frame}

%------------------------------------------------
\begin{frame}{Problème mathématique étudié}
On considère l’équation différentielle suivante :
\[
y'(t) = y(t), \quad y(0) = 1
\]

\vspace{0.3cm}
\textbf{Solution exacte :}
\[
y(t) = e^t
\]

\vspace{0.3cm}
L’erreur numérique est définie par :
\[
\text{Erreur}(t) = |y_{\text{exact}}(t) - y_{\text{num}}(t)|
\]
\end{frame}

%------------------------------------------------
\begin{frame}{Méthodes numériques utilisées}
\begin{itemize}
    \item \textbf{Méthode d’Euler}
    \begin{itemize}
        \item Méthode explicite du premier ordre
        \item Rapide mais peu précise
    \end{itemize}

    \item \textbf{Méthode de Heun}
    \begin{itemize}
        \item Méthode du second ordre
        \item Amélioration d’Euler par correction
    \end{itemize}

    \item \textbf{Méthode de Runge-Kutta d’ordre 4}
    \begin{itemize}
        \item Méthode d’ordre élevé
        \item Très bonne précision
    \end{itemize}
\end{itemize}
\end{frame}

%------------------------------------------------
\begin{frame}{Résultats numériques – Pas h = 0.4}
\begin{columns}
\column{0.5\textwidth}
\includegraphics[width=\linewidth]{graph_erreur pas egale 0,4.png}

\column{0.5\textwidth}
\includegraphics[width=\linewidth]{graph_temps pour pas egale 0,4.png}
\end{columns}

\vspace{0.2cm}
\textbf{Observation :}  
Les erreurs sont importantes pour Euler, tandis que RK4 reste très précis.
\end{frame}

%------------------------------------------------
\begin{frame}{Résultats numériques – Pas h = 0.2}
\begin{columns}
\column{0.5\textwidth}
\includegraphics[width=\linewidth]{graph_erreur pas egale 0,2.png}

\column{0.5\textwidth}
\includegraphics[width=\linewidth]{graph_temps pour pas egale 0,2.png}
\end{columns}

\vspace{0.2cm}
\textbf{Observation :}  
La réduction du pas améliore significativement la précision, surtout pour Heun.
\end{frame}

%------------------------------------------------
\begin{frame}{Résultats numériques – Pas h = 0.075}
\begin{columns}
\column{0.5\textwidth}
\includegraphics[width=\linewidth]{graph_erreur pas egale 0,075.png}

\column{0.5\textwidth}
\includegraphics[width=\linewidth]{graph_temps pour pas egale 0,075.png}
\end{columns}

\vspace{0.2cm}
\textbf{Observation :}  
RK4 présente une erreur quasi nulle, au prix d’un temps de calcul plus élevé.
\end{frame}

%------------------------------------------------
\begin{frame}{Conclusion générale}
\begin{itemize}
    \item La méthode d’Euler est rapide mais peu précise
    \item Heun offre un bon compromis précision/coût
    \item RK4 est la plus précise, surtout pour les petits pas
    \item Le pas de temps influence fortement l’erreur numérique
\end{itemize}

\vspace{0.3cm}
\textbf{Conclusion :}  
Le choix de la méthode dépend du compromis souhaité entre précision et temps de calcul.
\end{frame}

\end{document}
