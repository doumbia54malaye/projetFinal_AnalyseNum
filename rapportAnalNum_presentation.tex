\documentclass[12pt]{article}

\usepackage[utf8]{inputenc}
\usepackage[T1]{fontenc}
\usepackage[french]{babel}
\usepackage{amsmath, amssymb}
\usepackage{graphicx}
\usepackage{geometry}
\usepackage[hidelinks]{hyperref}
\usepackage{float}
\usepackage{fancyhdr}
\usepackage{xcolor}

\geometry{
    paperwidth=21cm,
    paperheight=18cm,
    margin=2.5cm
}

% Configuration du pied de page
\pagestyle{fancy}
\fancyhf{} % Efface tous les en-têtes et pieds de page
\fancyhead[L]{} % En-tête gauche vide
\fancyhead[C]{} % En-tête centre vide
\fancyhead[R]{} % En-tête droit vide
\fancyfoot[L]{\small Présentateur: DOUMBIA MALAYE}
\fancyfoot[C]{\small Analyse Numérique}
\fancyfoot[R]{\small Enseignant: Dr ZEZE | Page \thepage}
\renewcommand{\headrulewidth}{0pt} % Supprime la ligne d'en-tête
\renewcommand{\footrulewidth}{0.4pt} % Ajoute une ligne au pied de page

% Style pour la première page (page de titre)
\fancypagestyle{plain}{
    \fancyhf{}
    \fancyfoot[L]{\small Présentateur: DOUMBIA MALAYE}
    \fancyfoot[C]{\small Analyse Numérique}
    \fancyfoot[R]{\small Enseignant: Dr ZEZE | Page \thepage}
    \renewcommand{\headrulewidth}{0pt}
    \renewcommand{\footrulewidth}{0.4pt}
}

\title{\textbf{Comparaison des méthodes de quadrature numérique :\\
Gauss--Legendre, Gauss--Laguerre, Gauss--Tchebychev, Simpson et Spline quadratique\\
en termes de précision et de temps d'exécution}}
\author{DOUMBIA MALAYE}
\date{Année universitaire 2025-2026}

\begin{document}

\maketitle
\thispagestyle{plain} % Applique le style avec pied de page à la page de titre

\tableofcontents
\newpage

%=====================================================
\section{Introduction}

Dans le domaine du calcul scientifique, l'évaluation numérique des intégrales est un problème fondamental lorsque l'intégrale exacte d'une fonction ne peut pas être déterminée analytiquement. Plusieurs méthodes d'intégration numérique ont été développées à cet effet, parmi lesquelles figurent les méthodes de quadrature de Gauss, la méthode de Simpson ainsi que les méthodes basées sur l'interpolation comme les splines.

L'objectif de ce travail est de comparer cinq méthodes numériques :
\begin{itemize}
    \item Gauss--Legendre,
    \item Gauss--Laguerre,
    \item Gauss--Tchebychev,
    \item Simpson,
    \item Spline quadratique,
\end{itemize}

en termes de \textbf{précision (erreur)} et de \textbf{temps d'exécution}, sur quatre types de fonctions représentatives :
\begin{enumerate}
    \item Une fonction de la famille de Gauss--Laguerre,
    \item Une fonction de la famille de Gauss--Tchebychev,
    \item Une fonction combinant plusieurs comportements,
    \item Une fonction quelconque.
\end{enumerate}

Les comparaisons sont effectuées à l'aide de graphes représentant l'évolution de l'erreur et du temps d'exécution en fonction du paramètre de discrétisation \( n \).

%=====================================================
\section{Rappel théorique des méthodes}

\subsection{Méthode de Gauss--Legendre}

La méthode de Gauss--Legendre permet d'approximer une intégrale du type :
\[
\int_a^b f(x)\,dx
\]
par la formule :
\[
\int_a^b f(x)\,dx \approx \frac{b-a}{2} \sum_{i=1}^{n} w_i f(t_i)
\]
où \(t_i\) sont les nœuds et \(w_i\) les poids issus des polynômes de Legendre. Cette méthode est très précise pour les fonctions régulières.

\subsection{Méthode de Gauss--Laguerre}

Elle est destinée aux intégrales sur l'intervalle \([0,+\infty[\) de la forme :
\[
\int_0^{+\infty} g(x)e^{-x}\,dx
\]
Elle repose sur les polynômes de Laguerre. Dans ce travail, nous comparons la capacité de Gauss-Laguerre à intégrer sur \([0,+\infty[\) face aux autres méthodes appliquées sur un intervalle tronqué \([a,b]\).

\subsection{Méthode de Gauss--Tchebychev}

Elle permet d'approximer les intégrales de la forme :
\[
\int_{-1}^{1} \frac{h(x)}{\sqrt{1-x^2}}\,dx
\]
Les nœuds sont donnés par une formule trigonométrique simple et les poids sont constants. Comme pour Laguerre, un changement de variable permet son application à un intervalle quelconque.

\subsection{Méthode de Simpson}

La méthode de Simpson est une méthode d'intégration numérique fondée sur l'approximation de la fonction par des polynômes du second degré sur chaque sous-intervalle. La formule composite s'écrit :
\[
\int_a^b f(x)\,dx \approx \frac{h}{3}
\left[
f(a)+f(b)+4\sum f(x_{2k-1})+2\sum f(x_{2k})
\right]
\]
où \(h = \frac{b-a}{n}\) et \(n\) est pair.

\subsection{Spline quadratique}

Une spline quadratique est une fonction définie par morceaux, chaque morceau étant un polynôme de degré 2. Elle est généralement utilisée pour l'approximation et l'interpolation de fonctions.  

Dans ce travail, la spline quadratique est utilisée de manière indirecte comme \textbf{méthode d'intégration numérique}, en intégrant analytiquement chaque polynôme de la spline :
\[
\int_a^b f(x)\,dx \approx \int_a^b S(x)\,dx
\]
où \(S(x)\) est la spline interpolante de \(f(x)\).

%=====================================================
\section{Méthodologie}

\subsection{Outils utilisés}

\begin{itemize}
    \item Langage : Python
    \item Bibliothèques : NumPy, SymPy, Matplotlib
    \item Environnement : Jupyter Notebook (.ipynb)
\end{itemize}

\subsection{Principe de la comparaison}

Pour chaque fonction test :

\begin{enumerate}
    \item L'intégrale exacte est d'abord calculée avec SymPy.
    \item Chaque méthode numérique est appliquée pour plusieurs valeurs de \(n\).
    \item L'erreur est calculée par :
    \[
    E(n) = |I_{\text{exact}} - I_{\text{approx}}(n)|
    \]
    \item Le temps d'exécution est mesuré avec la fonction \texttt{time.perf\_counter()}.
    \item Deux graphes sont tracés :
    \begin{itemize}
        \item Erreur en fonction de \(n\),
        \item Temps d'exécution en fonction de \(n\).
    \end{itemize}
\end{enumerate}

\subsection{Fonctions tests utilisées}

\begin{enumerate}
    \item \textbf{Famille Gauss--Laguerre} :
    \[
    f(x) = (x^2 + 1)e^{-x}, \quad x \in [0,8]
    \]

    \item \textbf{Famille Gauss--Tchebychev} :
    \[
    f(x) = \frac{1}{\sqrt{1 - x^2}}, \quad x \in [-1,1]
    \]

    \item \textbf{Fonction combinée} :
    \[
    f(x) = \frac{(x^2 + 1)e^{-x}}{\sqrt{1 - x^2}}, \quad x \in [0,3]
    \]

    \item \textbf{Fonction quelconque} :
    \[
    f(x) = \sin(x) + x^2, \quad x \in [0,2]
    \]
\end{enumerate}

%=====================================================
\section{Présentation des résultats}

Pour chaque fonction test, deux graphiques ont été obtenus :
\begin{itemize}
    \item un graphe \textbf{Erreur en fonction de \(n\)} comportant cinq courbes, une par méthode ;
    \item un graphe \textbf{Temps d’exécution en fonction de \(n\)} comportant également cinq courbes.
\end{itemize}

Les figures correspondantes sont présentées ci-dessous.

Les valeurs \texttt{NaN} observées dans certains tableaux indiquent que la méthode de Gauss--Laguerre n’est pas directement applicable aux intégrales définies sur des intervalles finis, en l’absence d’une transformation adaptée du domaine d’intégration.

\begin{figure}[H]
\centering
\includegraphics[width=0.95\textwidth,     height=0.65\textheight]{tab_err_guerre.png}
\caption{Erreur en fonction de \(n\) -- Gauss Laguerre}
\end{figure}
\begin{figure}[H]
\centering
\includegraphics[width=0.95\textwidth,     height=0.65\textheight]{graph_err_guerre.png}
\caption{Erreur en fonction de \(n\) -- Gauss Laguerre}
\end{figure}

\begin{figure}[H]
\centering
\includegraphics[width=0.95\textwidth,     height=0.65\textheight]{graph_temps_guerre.png}
\caption{Temps en fonction de \(n\) -- Gauss Laguerre}
\end{figure}

\begin{figure}[H]
\centering
\includegraphics[width=0.95\textwidth,     height=0.65\textheight]{tab_err_tchebytchev.png}
\caption{Erreur en fonction de \(n\) -- Tchebytchev}
\end{figure}

\begin{figure}[H]
\centering
\includegraphics[width=0.95\textwidth,     height=0.65\textheight]{graph_err_tchebytchev.png}
\caption{Erreur en fonction de \(n\) -- Tchebytchev}
\end{figure}

\begin{figure}[H]
\centering
\includegraphics[width=0.95\textwidth,     height=0.65\textheight]{temp_temps-tchebytchev.png}
\caption{Erreur en fonction de \(n\) -- tchebytchev}
\end{figure}

\begin{figure}[H]
\centering
\includegraphics[width=0.95\textwidth,     height=0.65\textheight]{tab_err_combin.png}
\caption{Erreur en fonction de \(n\) -- Combinaison}
\end{figure}

\begin{figure}[H]
\centering
\includegraphics[width=0.95\textwidth,     height=0.65\textheight]{graph_err_combin.png}
\caption{Erreur en fonction de \(n\) -- Combinaison}
\end{figure}

\begin{figure}[H]
\centering
\includegraphics[width=0.95\textwidth,     height=0.65\textheight]{tab_err_quelconqu.png}
\caption{Erreur en fonction de \(n\) -- Quelconque}
\end{figure}
\begin{figure}[H]
\centering
\includegraphics[width=0.95\textwidth,
    height=0.65\textheight]{graph_err_quelconque.png}
\caption{Erreur en fonction de \(n\) -- Quelconque}
\end{figure}

\begin{figure}[H]
\centering
\includegraphics[width=0.95\textwidth,
    height=0.65\textheight]{graph_temps_quelconque.png}
\caption{Temps en fonction de \(n\) -- Quelconque}
\end{figure}

% Insérer de la même manière les figures des exemples 2, 3 et 4

%=====================================================
\section{Interprétation des résultats}

L’analyse conjointe des erreurs numériques et des temps d’exécution met en évidence
l’influence déterminante de l’adéquation entre la méthode d’intégration et la nature
de la fonction étudiée. Les résultats obtenus confirment les propriétés théoriques
des méthodes tout en illustrant leur comportement numérique pratique.

\section*{Analyse de la précision}

Pour les fonctions appartenant à une famille de quadrature donnée, les méthodes
associées atteignent une précision proche de la limite machine avec un nombre
réduit de points. C’est notamment le cas de la méthode de Gauss--Tchebychev pour
la fonction $\frac{1}{\sqrt{1-x^2}}$, pour laquelle l’erreur reste de l’ordre de
$10^{-16}$ quel que soit le nombre de points $n$. Ce comportement traduit une
exactitude quasi parfaite, conforme à la théorie.

Pour la fonction de type Gauss--Laguerre, la méthode de Gauss--Laguerre fournit
également une précision très élevée, proche de la précision machine. En revanche,
les méthodes définies sur des intervalles finis présentent une erreur qui ne diminue
pas significativement avec l’augmentation de $n$, en raison de la troncature de
l’intégrale impropre. L’erreur observée résulte alors à la fois de l’erreur de
quadrature et de l’erreur de troncature.

Dans le cas des fonctions combinées et des fonctions quelconques, définies sur des
intervalles finis et réguliers, la méthode de Gauss--Legendre se distingue par une
convergence très rapide. L’erreur décroît de manière spectaculaire et atteint
rapidement la limite de précision machine. Ce comportement confirme le caractère
spectral de la méthode de Gauss--Legendre pour les fonctions suffisamment régulières.

Les méthodes de Simpson et de la spline quadratique présentent une convergence plus
progressive. Leur précision s’améliore régulièrement avec l’augmentation du nombre
de points, mais elles nécessitent des discrétisations plus fines pour atteindre une
précision comparable à celle des méthodes de Gauss. La méthode de Gauss--Tchebychev,
lorsqu’elle est appliquée hors de son cadre théorique naturel, reste numériquement
stable mais présente une précision nettement inférieure, ce qui illustre
l’importance du choix du poids associé à la quadrature.

\section*{Analyse des temps d’exécution}

L’étude des temps d’exécution montre des comportements distincts selon les méthodes.
Les méthodes de quadrature de Gauss, et en particulier la méthode de Gauss--Legendre,
présentent un temps de calcul qui augmente avec le nombre de points, en raison du
coût associé à l’évaluation des nœuds et des poids ainsi qu’aux opérations de
transformation de l’intervalle. Cette augmentation est particulièrement visible
pour les fonctions régulières, où la précision élevée obtenue se fait au prix d’un
temps de calcul plus important.

La méthode de Gauss--Laguerre se révèle très efficace en termes de temps pour les
intégrales impropres adaptées, avec un coût de calcul faible et relativement stable
lorsque le nombre de points augmente. La méthode de Gauss--Tchebychev présente
également un temps d’exécution réduit, ce qui s’explique par la simplicité de ses
nœuds, définis explicitement par des formules trigonométriques.

Les méthodes classiques, telles que Simpson et la spline quadratique, affichent des
temps d’exécution globalement plus faibles que Gauss--Legendre pour des valeurs
modérées de $n$. Toutefois, la méthode de la spline quadratique devient légèrement
plus coûteuse lorsque le nombre de points augmente, en raison du calcul préalable
des coefficients de la spline.

\subsection*{Synthèse}

Ces résultats montrent qu’aucune méthode n’est universellement optimale. Les
méthodes de quadrature de Gauss offrent une précision exceptionnelle lorsqu’elles
sont appliquées dans leur cadre théorique naturel, mais peuvent devenir plus
coûteuses en temps de calcul. Les méthodes classiques restent attractives pour leur
simplicité et leur efficacité sur des fonctions régulières, bien qu’elles
nécessitent des discrétisations plus fines pour atteindre une précision élevée.
Enfin, l’application de certaines méthodes hors de leur domaine de validité
théorique permet de mieux comprendre leurs limites et souligne l’importance d’un
choix judicieux de la méthode d’intégration en fonction du problème traité.


%=====================================================
\section{Conclusion}

Dans ce travail, nous avons comparé cinq méthodes d'intégration numérique sur quatre types de fonctions représentatives.  

Les résultats obtenus montrent que :

\begin{itemize}
    \item Les méthodes de Gauss sont les plus rapides et les plus précises.
    \item La méthode de Simpson reste une excellente alternative pour un usage général.
    \item La spline quadratique, bien que n'étant pas à l'origine une méthode d'intégration, peut être utilisée efficacement pour approximer une intégrale via l'interpolation.
\end{itemize}

Ce travail met en évidence l'importance du choix de la méthode numérique en fonction de la nature de la fonction à intégrer et des contraintes de précision et de temps de calcul.

%=====================================================
\section*{Annexe}
\addcontentsline{toc}{section}{Annexe}

\begin{itemize}
    \item Codes Python complets dans le notebook.
    \item Tableaux de valeurs numériques.
    \item Graphes Erreur / Temps.
\end{itemize}

\end{document}