\documentclass{beamer}

\usetheme{Madrid}
\usecolortheme{default}

%================================================
% PAGE DE GARDE
%================================================
\title[Analyse Numérique]{Analyse Numérique\\
Équations différentielles et intégration numérique}
\author{DOUMBIA MALAYE}
\institute{
Discipline : Analyse Numérique\\
Enseignant : Dr ZEZE
}
\date{\today}

\begin{document}

%------------------------------------------------
\begin{frame}
\titlepage
\end{frame}

%================================================
% INTRODUCTION GÉNÉRALE
%================================================

\begin{frame}{Introduction générale}
L’analyse numérique permet de résoudre ou d’approximer des problèmes
mathématiques lorsque les solutions exactes sont difficiles à exploiter.

\vspace{0.3cm}
Ce travail est structuré en deux parties :
\begin{itemize}
\item la résolution numérique des équations différentielles ordinaires
\item la comparaison de méthodes d’intégration numérique
\end{itemize}
\end{frame}

%================================================
% PARTIE I : ÉDO
%================================================

\begin{frame}{Partie I : Équations différentielles ordinaires}
\centering
\textbf{Comparaison des méthodes d’Euler, Heun et Runge--Kutta d’ordre 4}
\end{frame}

%------------------------------------------------
\begin{frame}{Problème mathématique étudié}
On considère l’équation différentielle :
\[
y'(t) = y(t), \quad y(0) = 1
\]

La solution exacte est :
\[
y(t) = e^t
\]

L’erreur numérique est définie par :
\[
\text{Erreur}(t) = |y_{\text{exact}}(t) - y_{\text{num}}(t)|
\]
\end{frame}

%------------------------------------------------
\begin{frame}{Méthodes numériques utilisées (EDO)}
\begin{itemize}
\item \textbf{Euler} : méthode explicite d’ordre 1, rapide mais peu précise
\item \textbf{Heun} : méthode d’ordre 2, amélioration d’Euler
\item \textbf{RK4} : méthode d’ordre 4, très précise
\end{itemize}
\end{frame}

%------------------------------------------------
\begin{frame}{Résultats numériques — Pas $h = 0.4$}

\begin{columns}
\column{0.5\textwidth}
\centering
\includegraphics[width=\linewidth]{graph_erreur pas egale 0,4.png}
\small Erreur numérique

\column{0.5\textwidth}
\centering
\includegraphics[width=\linewidth]{graph_temps pour pas egale 0,4.png}
\small Temps d'exécution
\end{columns}

\vspace{0.2cm}
\textbf{Analyse :}
\begin{itemize}
\item Euler présente une erreur importante.
\item Heun améliore légèrement la précision.
\item RK4 reste très précis mais plus coûteux.
\end{itemize}
\end{frame}

%------------------------------------------------
\begin{frame}{Résultats numériques — Pas $h = 0.2$}

\begin{columns}
\column{0.5\textwidth}
\centering
\includegraphics[width=\linewidth]{graph_erreur pas egale 0,2.png}
\small Erreur numérique

\column{0.5\textwidth}
\centering
\includegraphics[width=\linewidth]{graph_temps pour pas egale 0,2.png}
\small Temps d'exécution
\end{columns}

\vspace{0.2cm}
\textbf{Analyse :}
\begin{itemize}
\item La réduction du pas améliore la précision globale.
\item Heun offre un bon compromis précision / temps.
\item RK4 atteint une erreur très faible.
\end{itemize}
\end{frame}

%------------------------------------------------
\begin{frame}{Résultats numériques — Pas $h = 0.075$}

\begin{columns}
\column{0.5\textwidth}
\centering
\includegraphics[width=\linewidth]{graph_erreur pas egale 0,075.png}
\small Erreur numérique

\column{0.5\textwidth}
\centering
\includegraphics[width=\linewidth]{graph_temps pour pas egale 0,075.png}
\small Temps d'exécution
\end{columns}

\vspace{0.2cm}
\textbf{Analyse :}
\begin{itemize}
\item RK4 atteint une précision quasi machine.
\item Euler reste peu précis malgré le petit pas.
\item Le coût temporel de RK4 augmente fortement.
\end{itemize}
\end{frame}

%------------------------------------------------
\begin{frame}{Conclusion partielle — EDO}
\begin{itemize}
\item Euler est rapide mais peu précis.
\item Heun représente un bon compromis.
\item RK4 est la méthode la plus précise.
\item Le pas de temps influence fortement l’erreur.
\end{itemize}
\end{frame}

%================================================
% PARTIE II : INTÉGRATION NUMÉRIQUE
%================================================

\begin{frame}{Partie II : Intégration numérique}
\centering
\textbf{Comparaison des méthodes d’intégration numérique}
\end{frame}

%------------------------------------------------
\begin{frame}{Objectifs de l’intégration numérique}
\begin{itemize}
\item Approcher des intégrales difficiles analytiquement
\item Comparer précision et temps d’exécution
\item Étudier l’adéquation méthode / fonction
\end{itemize}
\end{frame}

%------------------------------------------------
\begin{frame}{Méthodes étudiées}
\begin{itemize}
\item Gauss--Legendre
\item Gauss--Laguerre
\item Gauss--Tchebychev
\item Simpson
\item Splines quadratiques
\end{itemize}
\end{frame}

%------------------------------------------------
\begin{frame}{Rappels théoriques}
\begin{itemize}
\item Les méthodes de Gauss sont très précises
\item Les méthodes composites sont rapides
\item Le choix dépend du domaine et de la fonction
\end{itemize}
\end{frame}

%------------------------------------------------
\begin{frame}{Erreur numérique : Laguerre}
\centering
\includegraphics[width=0.9\textwidth]{graph_err_guerre.png}
\end{frame}

%------------------------------------------------
\begin{frame}{Temps d'exécution : Laguerre}
\centering
\includegraphics[width=0.9\textwidth]{graph_temps_guerre.png}
\end{frame}

%------------------------------------------------
\begin{frame}{Erreur numérique : Tchebychev}
\centering
\includegraphics[width=0.9\textwidth]{graph_err_tchebytchev.png}
\end{frame}

%------------------------------------------------
\begin{frame}{Temps d'exécution : Tchebychev}
\centering
\includegraphics[width=0.9\textwidth]{graph_temps_tchebychev.png}
\end{frame}

%------------------------------------------------
\begin{frame}{Erreur numérique : Fonction combinaison}
\centering
\includegraphics[width=0.9\textwidth]{graph_err_combin.png}
\end{frame}

%------------------------------------------------
\begin{frame}{Temps d'exécution : Fonction combinaison}
\centering
\includegraphics[width=0.9\textwidth]{graph_temps_combin.png}
\end{frame}

%------------------------------------------------
\begin{frame}{Erreur numérique : Fonction quelconque}
\centering
\includegraphics[width=0.9\textwidth]{graph_err_quelconque.png}
\end{frame}

%------------------------------------------------
\begin{frame}{Temps d'exécution : Fonction quelconque}
\centering
\includegraphics[width=0.9\textwidth]{graph_temps_quelconque.png}
\end{frame}

%================================================
% CONCLUSION GÉNÉRALE
%================================================

\begin{frame}{Conclusion générale}
\begin{itemize}
\item Les deux parties du projet mettent en évidence le compromis fondamental entre précision et coût de calcul.
\item Dans la résolution des EDO, les méthodes d’ordre élevé offrent une meilleure précision au prix d’un temps d’exécution plus important.
\item En intégration numérique, l’efficacité d’une méthode dépend fortement de l’adéquation entre la fonction, le domaine et la méthode choisie.
\item Ainsi, aucun algorithme n’est universel : le choix de la méthode doit toujours être guidé par le problème étudié.
\item L’analyse numérique constitue donc un outil essentiel pour résoudre efficacement des problèmes réels.
\end{itemize}
\end{frame}

\end{document}
